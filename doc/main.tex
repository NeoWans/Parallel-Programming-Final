\documentclass[a4paper]{article}

\input{style/ch_xelatex.tex}
\input{style/scala.tex}

%代码段设置
\lstset{numbers=left,
basicstyle=\tiny,
numberstyle=\tiny,
keywordstyle=\color{blue!70},
commentstyle=\color{red!50!green!50!blue!50},
frame=single, rulesepcolor=\color{red!20!green!20!blue!20},
escapeinside=``
}

\graphicspath{ {figures/} }
\usepackage{ctex}
\setCJKmainfont[ItalicFont=Noto Sans CJK SC Bold, BoldFont=Noto Serif CJK SC Black]{Noto Serif CJK SC}
\usepackage{graphicx}
\usepackage{color,framed}%文本框
\usepackage{listings}
\usepackage{caption}
\usepackage{amssymb}
\usepackage{enumerate}
\usepackage{xcolor}
\usepackage{bm} 
\usepackage{lastpage}%获得总页数
\usepackage{fancyhdr}
\usepackage{tabularx}  
\usepackage{geometry}
\usepackage{minted}
\usepackage{graphics}
\usepackage{subfigure}
\usepackage{float}
\usepackage{pdfpages}
\usepackage{pgfplots}
\pgfplotsset{width=10cm,compat=1.9}
\usepackage{multirow}
\usepackage{footnote}
\usepackage{booktabs}

%-----------------------伪代码------------------
\usepackage{algorithm}  
\usepackage{algorithmicx}  
\usepackage{algpseudocode}  
\floatname{algorithm}{Algorithm}  
\renewcommand{\algorithmicrequire}{\textbf{Input:}}  
\renewcommand{\algorithmicensure}{\textbf{Output:}} 
\usepackage{lipsum}  
\makeatletter
\newenvironment{breakablealgorithm}
  {% \begin{breakablealgorithm}
  \begin{center}
     \refstepcounter{algorithm}% New algorithm
     \hrule height.8pt depth0pt \kern2pt% \@fs@pre for \@fs@ruled
     \renewcommand{\caption}[2][\relax]{% Make a new \caption
      {\raggedright\textbf{\ALG@name~\thealgorithm} ##2\par}%
      \ifx\relax##1\relax % #1 is \relax
         \addcontentsline{loa}{algorithm}{\protect\numberline{\thealgorithm}##2}%
      \else % #1 is not \relax
         \addcontentsline{loa}{algorithm}{\protect\numberline{\thealgorithm}##1}%
      \fi
      \kern2pt\hrule\kern2pt
     }
  }{% \end{breakablealgorithm}
     \kern2pt\hrule\relax% \@fs@post for \@fs@ruled
  \end{center}
  }
\makeatother
%------------------------代码-------------------
\usepackage{xcolor} 
\usepackage{listings} 
\lstset{ 
breaklines,%自动换行
basicstyle=\small,
escapeinside=``,
keywordstyle=\color{ blue!70} \bfseries,
commentstyle=\color{red!50!green!50!blue!50},% 
stringstyle=\ttfamily,% 
extendedchars=false,% 
linewidth=\textwidth,% 
numbers=left,% 
numberstyle=\tiny \color{blue!50},% 
frame=trbl% 
rulesepcolor= \color{ red!20!green!20!blue!20} 
}

%-------------------------页面边距--------------
\geometry{a4paper,left=2.3cm,right=2.3cm,top=2.7cm,bottom=2.7cm}
%-------------------------页眉页脚--------------
\usepackage{fancyhdr}
\pagestyle{fancy}
\lhead{\kaishu \leftmark}
% \chead{}
\rhead{\kaishu 并行程序设计期末研究报告}%加粗\bfseries 
\lfoot{}
\cfoot{\thepage}
\rfoot{}
\renewcommand{\headrulewidth}{0.1pt}  
\renewcommand{\footrulewidth}{0pt}%去掉横线
\newcommand{\HRule}{\rule{\linewidth}{0.5mm}}%标题横线
\newcommand{\HRulegrossa}{\rule{\linewidth}{1.2mm}}
\setlength{\textfloatsep}{10mm}%设置图片的前后间距
%--------------------文档内容--------------------

\begin{document}
\renewcommand{\contentsname}{目\ 录}
\renewcommand{\appendixname}{附录}
\renewcommand{\appendixpagename}{附录}
\renewcommand{\refname}{参考文献}
\renewcommand{\figurename}{图}
\renewcommand{\tablename}{表}
\renewcommand{\today}{\number\year 年 \number\month 月 \number\day 日}

%-------------------------封面----------------
\begin{titlepage}
  \begin{center}
    \includegraphics[width=0.8\textwidth]{NKU.png}\\[1cm]
    \vspace{20mm}
    \textbf{\huge\textbf{\kaishu{计算机学院}}}\\[0.5cm]
    \textbf{\huge{\kaishu{并行程序设计}}}\\[2.3cm]
    \textbf{\Huge\textbf{\kaishu{特殊高斯消去法的并行优化}}}

    \vspace{\fill}

    \textbf{\Large \textbf{并行程序设计期末研究报告}}\\[0.8cm]
    \HRule \\[0.9cm]
    \HRule \\[2.0cm]
    \centering
    \textsc{\LARGE \kaishu{姓名\ :\ 丁屹、卢麒萱}}\\[0.5cm]
    \textsc{\LARGE \kaishu{学号\ :\ 2013280、2010519}}\\[0.5cm]
    \textsc{\LARGE \kaishu{专业\ :\ 计算机科学与技术}}\\[0.5cm]

    \vfill
    {\Large \today}
  \end{center}
\end{titlepage}

\renewcommand {\thefigure}{\thesection{}.\arabic{figure}}%图片按章标号
\renewcommand{\figurename}{图}
\renewcommand{\contentsname}{目录}
\cfoot{\thepage\ of \pageref{LastPage}}%当前页 of 总页数


% 生成目录
\clearpage
\tableofcontents
\newpage

\section{问题描述}
普通高斯消去的计算模式如图 \ref{gauss} 所示,在第$k$步时,对第$k$行从$(k, k)$开始进行除法操作,并且将后续的$k + 1$至$N$行进行减去第$k$行的操作,串行算法如下面伪代码所示。
\begin{figure}
  \centering
  \includegraphics[width=1\textwidth]{gauss.png}
  \caption{高斯消去法示意图}
  \label{gauss}
\end{figure}
\begin{breakablealgorithm}
  \caption{普通高斯消元算法伪代码}
  \begin{algorithmic}[1] %每行显示行号  
    \Function {LU}{}
    \For {$k:=0$\ \textbf{to}\ $n$}
    \For {$j:=k+1$\ \textbf{to}\ $n$}
    \State {$A[k,j]:=A[k,j]/A[k,k]$}
    \EndFor
    \State{$A[k,k]:=1.0$}
    \For {$i:=k+1$\ \textbf{to}\ $n$}
    \For {$j:=k+1$\ \textbf{to}\ $n$}
    \State {$A[i,j]:=A[i,j]-A[i,k]*A[k,j]$}
    \EndFor
    \State{$A[i,k]:=0$}
    \EndFor
    \EndFor
    \EndFunction
  \end{algorithmic}
\end{breakablealgorithm}

\section{研究设计}
源码链接:
\url{https://github.com/ArcanusNEO/Parallel-Programming/tree/master/8}

\subsection{测试用例}
测试用例由老师提供的Groebner.7z压缩包解压后获得,总共11组数据,软链接至res/目录下,命名规则为\%组号\%.0 (非零消元子)、\%组号\%.1 (被消元行)、\%组号\%.2 (消元结果)

\subsection{实验环境和相关配置}
实验在本地 x86 Arch Linux 环境下完成,使用 Makefile 构建项目,开启 Ofast 加速;
使用的 CPU 为 AMD Ryzen 8C16T,显卡为 nVIDIA RTX 2060。

\subsection{串行稀疏矩阵算法}
使用STL list存储矩阵中每行的非零位置,逐行放入嵌套的外层STL list;
使用STL map存储消元行首项与消元行的映射

\subsection{串行位元矩阵算法}


\section{算法分析}
\subsection{复杂度分析}


\subsection{运行时间分析}
由于华为鲲鹏服务器工作不稳定,所以只在本地 x86 环境实验。


\end{document}